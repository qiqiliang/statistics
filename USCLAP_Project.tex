% Options for packages loaded elsewhere
\PassOptionsToPackage{unicode}{hyperref}
\PassOptionsToPackage{hyphens}{url}
%
\documentclass[
]{article}
\usepackage{lmodern}
\usepackage{amssymb,amsmath}
\usepackage{ifxetex,ifluatex}
\ifnum 0\ifxetex 1\fi\ifluatex 1\fi=0 % if pdftex
  \usepackage[T1]{fontenc}
  \usepackage[utf8]{inputenc}
  \usepackage{textcomp} % provide euro and other symbols
\else % if luatex or xetex
  \usepackage{unicode-math}
  \defaultfontfeatures{Scale=MatchLowercase}
  \defaultfontfeatures[\rmfamily]{Ligatures=TeX,Scale=1}
\fi
% Use upquote if available, for straight quotes in verbatim environments
\IfFileExists{upquote.sty}{\usepackage{upquote}}{}
\IfFileExists{microtype.sty}{% use microtype if available
  \usepackage[]{microtype}
  \UseMicrotypeSet[protrusion]{basicmath} % disable protrusion for tt fonts
}{}
\makeatletter
\@ifundefined{KOMAClassName}{% if non-KOMA class
  \IfFileExists{parskip.sty}{%
    \usepackage{parskip}
  }{% else
    \setlength{\parindent}{0pt}
    \setlength{\parskip}{6pt plus 2pt minus 1pt}}
}{% if KOMA class
  \KOMAoptions{parskip=half}}
\makeatother
\usepackage{xcolor}
\IfFileExists{xurl.sty}{\usepackage{xurl}}{} % add URL line breaks if available
\IfFileExists{bookmark.sty}{\usepackage{bookmark}}{\usepackage{hyperref}}
\hypersetup{
  pdftitle={Impact of Opioid Prescription Rate on Suicide Rates and Poverty Levels of Counties in the United States},
  hidelinks,
  pdfcreator={LaTeX via pandoc}}
\urlstyle{same} % disable monospaced font for URLs
\usepackage[margin=1in]{geometry}
\usepackage{longtable,booktabs}
% Correct order of tables after \paragraph or \subparagraph
\usepackage{etoolbox}
\makeatletter
\patchcmd\longtable{\par}{\if@noskipsec\mbox{}\fi\par}{}{}
\makeatother
% Allow footnotes in longtable head/foot
\IfFileExists{footnotehyper.sty}{\usepackage{footnotehyper}}{\usepackage{footnote}}
\makesavenoteenv{longtable}
\usepackage{graphicx,grffile}
\makeatletter
\def\maxwidth{\ifdim\Gin@nat@width>\linewidth\linewidth\else\Gin@nat@width\fi}
\def\maxheight{\ifdim\Gin@nat@height>\textheight\textheight\else\Gin@nat@height\fi}
\makeatother
% Scale images if necessary, so that they will not overflow the page
% margins by default, and it is still possible to overwrite the defaults
% using explicit options in \includegraphics[width, height, ...]{}
\setkeys{Gin}{width=\maxwidth,height=\maxheight,keepaspectratio}
% Set default figure placement to htbp
\makeatletter
\def\fps@figure{htbp}
\makeatother
\setlength{\emergencystretch}{3em} % prevent overfull lines
\providecommand{\tightlist}{%
  \setlength{\itemsep}{0pt}\setlength{\parskip}{0pt}}
\setcounter{secnumdepth}{-\maxdimen} % remove section numbering

\title{Impact of Opioid Prescription Rate on Suicide Rates and Poverty Levels
of Counties in the United States}
\author{}
\date{\vspace{-2.5em}}

\begin{document}
\maketitle

\hypertarget{abstract}{%
\subsection{Abstract}\label{abstract}}

Opioids continue to flood into our communities through prescriptions
from doctors and dentists. In this study, we attempt to determine the
relationship between opioid prescription rates of a county and the
county's associated suicide rate and poverty levels. Data was pulled
from government websites and compiled into one large data set. The
opioid prescription rate was then used as the independent variable to be
compared against the other two variables in multiple statistical tests
to see if there is a positive correlation between the three variables.
All p-values were found to be well below our significance level of 0.05,
all slopes were positive, and all adjusted r2 values were positive as
well. Our results suggest that there is a positive relationship between
the three analyzed variables---\textbf{\emph{if opiod prescriptions are
fixed, suicide rates should go up with poverty levels}}. We hope that
this research project will bring greater social awareness to the
severity of the opioid crisis and thus prompt greater support and
resource distribution efforts at the community level.

\newpage

\hypertarget{introduction}{%
\subsection{Introduction}\label{introduction}}

On August 13th, 2020, NPR published ``Doctors and Dentists Still
Flooding U.S With Opioid Prescriptions,'' which discusses how health
care professionals are still prescribing opioids at an alarming rate
{[}3{]}. The opioid epidemic has plagued our country since the late
1990s, causing widespread misuse and fatal addictions. Furthermore,
according to an NIH article, ``Suicide Deaths are a Major Component of
the Opioid Crisis that Must be Addressed,'' some experts believe that up
to 30\% of opioid overdoses may actually be suicides {[}5{]}. In fact,
the strength of the relationship between suicides and opioids have
seemed to increase with time and in a 2017 study, it was shown that
people who misused prescription opioids had a much higher rate of
suicide ideation, even when other health conditions were controlled
{[}1{]}. Additionally, those misusing opioid prescriptions were about
twice as likely to attempt suicide compared to those that did not misuse
opioids {[}1{]}. Per these articles and the ongoing opioid crisis, this
project looks into the factors that may affect the rate of opioid
prescriptions in relation to suicides rates and poverty levels.
Specifically, this project attempts to address the question: Is a larger
rate of legal opioid prescription related to higher numbers in suicide
and poverty rates? Our null hypothesis assumes there is no correlation
between opioid prescription rates, suicide rates, and poverty levels.
Our one-sided alternate hypothesis checks for a positive correlation
between the three variables.

\hypertarget{methods}{%
\subsection{Methods}\label{methods}}

We used three separate datasets in this project. From the Center for
Disease Control and Prevention (CDC) website, we obtained an opioid
prescription dataset that provides the number of opioid prescriptions
per 100 residents, by state and county, from 2005 to 2015 {[}6{]}. We
also obtained the suicide dataset is from the CDC and provides the
suicide death rates in the United States per county per 100,000 people
from 2008 to 2014 {[}4{]}. Finally, the poverty dataset is obtained from
the Health Resources and Services Administration (HRSA) website. The
data includes the percentage of families, by state and county, with
incomes below 1.0, 1.5, and 2.0 times the Federal Poverty Level (FPL)
from 2014 to 2018 {[}2{]}. The independent variable for this project is
opioid prescription rates (OPR). The dependent variables are suicide
rates (SR), poverty index (PI), and the percentages of families below 1
times the Federal Poverty Rate (P1.0), 1.5 times the FPL (P1.5), and 2.0
times the FPL (P2.0). The OPR describes the average opioid prescriptions
dispensed per 100 persons in each county. The SR describes death rates
(by suicide) per 100,000 people in each county. The P1.0, P1.5, and P2.0
describes the average percentage of families with incomes below 1.0,
1.5, and 2.0 times the FPL, inclusive and grouped by county. The PI is a
weighted percentage of people in poverty. This calculated value prevents
double counting of people and weights those below the FPL more heavily
in each county. This was calculated by taking P1.0 with full weight,
P1.50 with 2/3 weight, P2.00 with 1/3 weight, and families above P2.00
with no weight. The higher the poverty index, the more families near or
in poverty in respective counties. The population in our study is the
counties in the U.S., not including territories and provided there is
data for all three variables available (\(n = 1768\)). To conduct our
research, we first plotted scatter plots, residual plots, and gathered
our test statistic, the slope (\(beta_1\)), along with \(R^2\), and
p-values from simple regression models: SR, P1.0, P1.5, P2.0, or PI as a
function of OPR. Simulations were run for all simple regression
models---null permutations of the models (assume no relationship between
the independent and dependent variables), null plots with test statistic
(see Appendix for more), null p-values, bootstrap simulations to
generate 95\% confidence intervals to show the range in which the test
statistic could have been in. After running the models, we made a
scatter plot incorporating OPR, SR, and PI to better visualize the
relationships between them.

\hypertarget{results}{%
\subsection{Results}\label{results}}

Our first question asks whether or not there is a relationship between
opioid prescription rates (OPR), suicide rates (SR), and the various
poverty statistics we used (P1.0, P1.5, P2.0, PI). If there is a
relationship, is it a positive trending relationship? From our linear
regression models, we found that OPR versus all other variables yielded
p-values \textless{} 2.26e-16, small, positive slopes and adjusted
\(R^2\) values (Table 1).

\begin{longtable}[]{@{}lllllll@{}}
\toprule
\begin{minipage}[b]{0.12\columnwidth}\raggedright
Model\strut
\end{minipage} & \begin{minipage}[b]{0.12\columnwidth}\raggedright
Null Distribution P-Value\strut
\end{minipage} & \begin{minipage}[b]{0.12\columnwidth}\raggedright
LM P-Value\strut
\end{minipage} & \begin{minipage}[b]{0.12\columnwidth}\raggedright
Adjusted \(R^2\)\strut
\end{minipage} & \begin{minipage}[b]{0.12\columnwidth}\raggedright
Slope\strut
\end{minipage} & \begin{minipage}[b]{0.12\columnwidth}\raggedright
LM Slope CI\strut
\end{minipage} & \begin{minipage}[b]{0.12\columnwidth}\raggedright
Bootstrap CI\strut
\end{minipage}\tabularnewline
\midrule
\endhead
\begin{minipage}[t]{0.12\columnwidth}\raggedright
OPR x SR\strut
\end{minipage} & \begin{minipage}[t]{0.12\columnwidth}\raggedright
0\strut
\end{minipage} & \begin{minipage}[t]{0.12\columnwidth}\raggedright
\textless{} 2.2e-16\strut
\end{minipage} & \begin{minipage}[t]{0.12\columnwidth}\raggedright
0.06084\strut
\end{minipage} & \begin{minipage}[t]{0.12\columnwidth}\raggedright
0.031532\strut
\end{minipage} & \begin{minipage}[t]{0.12\columnwidth}\raggedright
(0.026, 0.037)\strut
\end{minipage} & \begin{minipage}[t]{0.12\columnwidth}\raggedright
(0.02573316, 0.03740757)\strut
\end{minipage}\tabularnewline
\begin{minipage}[t]{0.12\columnwidth}\raggedright
OPR x P1.0\strut
\end{minipage} & \begin{minipage}[t]{0.12\columnwidth}\raggedright
0\strut
\end{minipage} & \begin{minipage}[t]{0.12\columnwidth}\raggedright
\textless{} 2.2e-16\strut
\end{minipage} & \begin{minipage}[t]{0.12\columnwidth}\raggedright
0.1608\strut
\end{minipage} & \begin{minipage}[t]{0.12\columnwidth}\raggedright
0.044857\strut
\end{minipage} & \begin{minipage}[t]{0.12\columnwidth}\raggedright
(0.040, 0.050)\strut
\end{minipage} & \begin{minipage}[t]{0.12\columnwidth}\raggedright
(0.03914246, 0.05022237)\strut
\end{minipage}\tabularnewline
\begin{minipage}[t]{0.12\columnwidth}\raggedright
OPR x P1.5\strut
\end{minipage} & \begin{minipage}[t]{0.12\columnwidth}\raggedright
0\strut
\end{minipage} & \begin{minipage}[t]{0.12\columnwidth}\raggedright
\textless{} 2.2e-16\strut
\end{minipage} & \begin{minipage}[t]{0.12\columnwidth}\raggedright
0.159\strut
\end{minipage} & \begin{minipage}[t]{0.12\columnwidth}\raggedright
0.067001\strut
\end{minipage} & \begin{minipage}[t]{0.12\columnwidth}\raggedright
(0.060. 0.074)\strut
\end{minipage} & \begin{minipage}[t]{0.12\columnwidth}\raggedright
(0.05913409, 0.07493694)\strut
\end{minipage}\tabularnewline
\begin{minipage}[t]{0.12\columnwidth}\raggedright
OPR x P2.0\strut
\end{minipage} & \begin{minipage}[t]{0.12\columnwidth}\raggedright
0\strut
\end{minipage} & \begin{minipage}[t]{0.12\columnwidth}\raggedright
\textless{} 2.2e-16\strut
\end{minipage} & \begin{minipage}[t]{0.12\columnwidth}\raggedright
0.1632\strut
\end{minipage} & \begin{minipage}[t]{0.12\columnwidth}\raggedright
0.086358\strut
\end{minipage} & \begin{minipage}[t]{0.12\columnwidth}\raggedright
(0.077, 0.095)\strut
\end{minipage} & \begin{minipage}[t]{0.12\columnwidth}\raggedright
(0.07714104, 0.09526279)\strut
\end{minipage}\tabularnewline
\begin{minipage}[t]{0.12\columnwidth}\raggedright
OPR x PI\strut
\end{minipage} & \begin{minipage}[t]{0.12\columnwidth}\raggedright
0\strut
\end{minipage} & \begin{minipage}[t]{0.12\columnwidth}\raggedright
\textless{} 2.2e-16\strut
\end{minipage} & \begin{minipage}[t]{0.12\columnwidth}\raggedright
0.1652\strut
\end{minipage} & \begin{minipage}[t]{0.12\columnwidth}\raggedright
0.066072\strut
\end{minipage} & \begin{minipage}[t]{0.12\columnwidth}\raggedright
(0.059, 0.073)\strut
\end{minipage} & \begin{minipage}[t]{0.12\columnwidth}\raggedright
(0.05869566, 0.07321176)\strut
\end{minipage}\tabularnewline
\bottomrule
\end{longtable}

\end{document}
